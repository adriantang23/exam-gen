
\documentclass[letterpaper,11pt]{article}
\usepackage{xfrac}
\usepackage{fullpage,amsthm,amsmath,amsfonts,amssymb,graphicx,color,clrscode,enumitem,float}
\usepackage[hidelinks]{hyperref}
% Choose one option (bubbles)
\newcommand{\chooseone}{{\large{\mbox{$\bigcirc$}}\ }}

% For automata drawings
\usepackage{pgf,wrapfig}
\usepackage{tikz}
\usepackage{forest}
\usepackage{skak} %for drawing chess symbols
\usetikzlibrary{arrows,automata}%,snakes}
\usetikzlibrary{decorations.pathreplacing} %for braces
\addtolength{\textwidth}{0.2in}
\addtolength{\oddsidemargin}{-0.1in}
\addtolength{\evensidemargin}{-0.1in}

\addtolength{\textheight}{0.5in}
\addtolength{\topmargin}{-0.25in}

\theoremstyle{plain}% default
\newtheorem{thrm}{Theorem}[section]
\newtheorem{lemm}[thrm]{Lemma}
\newtheorem{prop}[thrm]{Proposition}
\newtheorem*{cor}{Corollary}
\newtheorem*{rem}{Remark}

\theoremstyle{definition}
\newtheorem{claim}{Claim}
\newtheorem{definition}{Definition}
\newtheorem{theorem}{Theorem}
\newtheorem{lemma}{Lemma}
\newtheorem{observation}{Observation}
\newtheorem{problem}{\textcolor{blue}{Problem}}
\newtheorem*{xproblem}{EXTRA Problem}
\newtheorem{proposition}{Proposition}

\newenvironment{solution}
  {\begin{quote}\color{blue}\textbf{Solution}. \small}
  {\end{quote}}


\theoremstyle{plain}% default


\begin{document}

\newcommand{\E}{\mathbb{E}}
\newcommand{\Var}{\mathrm{Var}}

{\noindent\large
CS 237: {\em Probability in Computing} \hfill Professors: Alina Ene, Tiago Januario\\
Boston University \hfill \today\\}
\vspace{1pt} \hrulefill\vspace{3mm}
\begin{center}
{\Large\bf Homework 10 -- Due Wednesday, November 20$^{rd}$, 2024 \underline{by 9:00 PM}}
\end{center}

\begin{itemize}
\item Provide step-by-step explanations, not just answers. Answers without explanations will earn a small fraction of the points.
\item Submit your solutions on Gradescope. Remember to include information about your collaborators (or say ``Collaborators: none'').
\end{itemize}

\begin{problem}[\textbf{Gachapon machine}, 10 points]
Riva likes playing gacha games. The gacha machine has 100 unique bunny characters. Each time Riva rolls the gacha machine, the machine outputs a bunny character uniformly at random with replacement, independent of other rolls.
    \begin{enumerate}[label=(\alph*)]
        \item What is the expected number rolls until Riva collects all 100 types of bunnies?
        \item What is the expected number of rolls until Riva collects exactly 50 types of unique bunnies?
        \item Riva rolls the machine 200 times. What is the expected number of unique bunny types she gets?
        \item Riva rolls the machine 300 times. What is the expected number of bunny types she gets exactly once?
        \item Suppose Riva Bun is one of the bunnies and Riva wants it. Use the Most Used Inequality, $1-x\leq e^{-x}$, to show that if she rolls the machine 100 times, the probability that she gets Riva Bun at least once is at least $1/2$. \textit{Hint: Remember that $e \geq 2$, so what does that tell you about the relation between $\frac 1 e$ and $\frac 1 2$?}
    \end{enumerate}
\end{problem}

\begin{problem}[\textbf{Random vacation}, 10 points] After a long period of work, Annie is excited to leave town for a vacation. At the same time, her pet bunny Ash is equally excited to start chewing on her stuff while she's gone. \textbf{For each part of this problem, first identify the correct distribution and its parameters.}

    \begin{enumerate}[label=(\alph*)]
        \item Being a former CS237 TA, Annie will roll a 20-sided die to determine the number of days she will spend on vacation. What is the expected number of days she will spend on vacation?
        \item It is currently Monday. Starting today, Annie will depart for her vacation on the first day that has ideal weather. If each day has a $0.3$ chance to have ideal weather independently of others, what is the probability Annie will depart on a weekend (\textbf{any} weekend, not just the first)?
        \item There are 12 valuable (but tasty!) items in the house that Ash might chew on. Being a very random rabbit, he will chew on each carpet with probability $0.4$, independently of other items. If each chewed item costs \$30 to repair, what is the variance of the number of dollars Annie will spend in repairs?
        \item When Annie gets home, she will give Ash a banana slice if he chewed on one or fewer of the 12 items. What is the expected value of the indicator that Ash gets a banana slice?
    \end{enumerate}
\end{problem}

\begin{problem} (\textbf{From Y to Y}, 10 points)  Alice wants to give her friend a string for their birthday. She knows that her friend likes randomness and dislikes repeating characters. Thus, Alice generates her strings in the following way:
    \begin{enumerate}[label=(\alph*)]
        \item Alice uniformly and randomly chooses a capital letter that is not Y to be the first letter of her string, and then she repeatedly adds a uniformly random capital letter (A - Z) that isn't the same as the latest character in the string until she adds the first Y to the string. What is the expected value and the variance of the total length of her string?
    \end{enumerate}

    Alice knows that her friend's favorite letter is Y, so she decides to generate her string in this way instead:
    \begin{enumerate}[label=(\alph*)]\setcounter{enumi}{1}
        \item Alice chooses the first character of the string to be Y, and then she repeatedly adds a uniformly random capital letter (A - Z) that isn't the same as the latest character in the string until she adds the second Y to the string. What is the expected value and the variance of the total length of her string?
    \end{enumerate}

    Alice decides to not think too hard about how to generate a string so she decides to generate her string in this way instead:
    \begin{enumerate}[label=(\alph*)]\setcounter{enumi}{2}
        \item Alice uniformly and randomly chooses a capital letter to be the first letter of her string, and then she repeatedly adds a uniformly random capital letter (A - Z) that isn't the same as the latest character in the string until she adds another Y to the string. What is the expected value and the variance of the total length of her string?
    \end{enumerate}
\end{problem}

\begin{problem}[\textbf{Jurnalism Mistaks}, 10 points] Phillip is a senior journalist who writes 6 articles every day; for each article he writes, he makes a mistake with probability 0.2 independently of other articles. Bozena is a trainee jornalist who writes 4 articles every day; for each article she writes, she makes a mistake with probability 0.3 independently of other articles.
\begin{enumerate}[label=(\alph*)]
    \item Find the standard deviation of the total number of mistakes they make in 5 days.
    
    \item The two journalists make a total of 2 mistakes on a given day. What is the probability that Philip made more mistakes than Bozena that day?
    
    \item The next day, at least one of the 10 articles they wrote contains a mistake. Find the expected number of total mistakes they made that day.
    
    {\em Hint:} Find the expected number of mistakes Philipp and Bozena make in a day. Then apply the Law of Total Expectation.


    
\end{enumerate}
\end{problem}

\begin{problem}[\textbf{Ping Pong Balls}, 10 points]
    10 ping pong balls are numbered from 1 to 10 and placed in a bag. A ping pong ball is randomly picked and is removed from the bag. Let $X$ be the number on the ping pong ball that is removed.

    \begin{enumerate} [label=(\alph*)]
        \item Define the distribution of $X$. What is the expected value of the number on the ping pong ball that is removed?
        
        \item What is the probability that the number on the drawn ping pong ball is between 7 and 9?
        
        \item Let $Y$ be the indicator random variable that takes the value 1 if the ping pong ball drawn has a number between 7 and 9, and 0 otherwise. Find the distribution, parameter, and PDF of $Y$. 
        
        \item Suppose Riva, Steve, Noah, and Champ each randomly pick a ping pong ball from the bag with replacement. What is the expected number of times a ping pong ball numbered between 7 and 9 is picked?
        
    \end{enumerate}
\end{problem}

\begin{problem}[{\bf No programming assignment this week}, 0 points]
Happy Thanksgiving!
\end{problem}

\end{document}